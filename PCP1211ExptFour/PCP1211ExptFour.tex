\documentclass[a4paper,10pt]{article}
\usepackage[lmargin=2.0cm, rmargin=1.0cm,tmargin=3.5cm,bmargin=1.5cm]{geometry}
\usepackage{color,graphics}
\usepackage[export]{adjustbox}
\usepackage{lipsum}
\usepackage{comment}
\usepackage{graphicx}

\usepackage{listings}
\usepackage[scaled=0.75]{helvet}

\begin{document}
\setcounter{secnumdepth}{-1} 

\begin{center}
\textbf{\LARGE Running Hadoop On Ubuntu Linux (Multi-Node Cluster)}
\end{center}

\raggedright Expt No: 4 \hfill \raggedleft April  24, 2019 \\ 

\raggedright Author: Subalakshmi Shanthosi S  (186001008) \par 

\noindent\makebox[\linewidth]{\rule{\textwidth}{1pt}} 

\section{Aim}
To configure and install  Hadoop Multinode cluster in Ubuntu 16.04 LTS OS flavour.

\section{Software's Used}
\begin{itemize}
  \item Ubuntu  16.04 LTS
  \item Hadoop 1.0.3
\end{itemize}

\section{Description}
Installation of Oracle VirtualBox with guest Operating System as Ubuntu 16.04.Installation of neccessary packages namely - openssh-server,openssh-client,java,hadoop in the created virtualbox instance.
\section{Procedure}

\begin{enumerate}
	\item Launch Ubuntu 16.04 LTS.
	\item Login to the OS with sudo permission and install the following packages using apt-get command.
	\begin{itemize}
		\item openssh-server
		\item openssh-client
		\item java jdk 8
		\item javac compiler
                     \item hadoop 1.0.3
	\end{itemize}
	\item Create a new user with sudo permission (hduser:hadoop).
    \item Log into the hduser and do the following:
    \begin{itemize}
    	\item Copy the hadoop executable to /usr/local directory.
    	\item Install and configure appropriate environment variables and parameters in the following configuration files: 
    	\begin{itemize}
    		\item conf/hadoop-env.sh : Configure JAVA\_HOME and HADOOP\_HOME with appropriate values.
    		\item conf/core-site.xml : Configure hadoop default temp directory and default file system.
    		\item conf/mapred-site.xml : JobTracker name and port number.
    		\item conf/hdfs-site.xml : Default replication factor specification.
    	\end{itemize}
        \item Format the namenode by specified dfs.name.dir by running command : /usr/local/hadoop/bin/hadoop namenode -format .
        \item Starting the local hadoop single node cluster by running command: 
        /usr/local/hadoop/bin/start-all.sh .
        \item To check the current running Hadoop Processes by running command : 
        jps .
        \item Stopping the local hadoop single node cluster by running command : 
        /usr/local/hadoop/bin/stop-all.sh .
    \end{itemize}
\pagebreak
\end{enumerate}

\section{Output}
\begin{figure}[h]
	\includegraphics[scale=0.30,center]{exptFourScreenShot/fig1.png}
	\caption{Install openssh-server,openssh-client in Ubuntu OS.}
	\label{fig:1}
	
\end{figure}
\begin{figure}[h]
	\includegraphics[scale=0.34,center]{exptFourScreenShot/fig2.png}
	\caption{Setting Java Home environment variable to the specified download path of JDK-1.7.}
	\label{fig:2}
\end{figure}

\begin{figure}[h]
	\includegraphics[scale=0.34,center]{exptFourScreenShot/fig3.png}
	\caption{Adding a dedicated hadoop system user.}
	\label{fig:3}
\end{figure}
\newpage
\begin{figure}[h]
	\includegraphics[scale=0.30,center]{exptFourScreenShot/fig4.png}
	\caption{Configuring SSH in newly created user.}
	\label{fig:4}
\end{figure}

\begin{figure}[h]
	\includegraphics[scale=0.34,center]{exptFourScreenShot/fig5.png}
	\caption{Disabling IPv6 in the newly created user account.}
	\label{fig:5.1}
\end{figure}

\begin{figure}[h]
	\includegraphics[scale=0.34,center]{exptFourScreenShot/fig5Two.png}
	\caption{Disabling IPv6 in the newly created user account.}
	\label{fig:5.2}
\end{figure}

\begin{figure}[h]
	\includegraphics[scale=0.45,center]{exptFourScreenShot/fig6.png}
	\caption{Installation of Hadoop 1.0.3 in new user login.}
	\label{fig:6}
\end{figure}
\newpage
\begin{figure}[h]
	\includegraphics[scale=0.30,center]{exptFourScreenShot/fig7.png}
	\caption{Configuring hadoop core-site.xml .}
	\label{fig:7}
\end{figure}

	\begin{figure}[h]
		\includegraphics[scale=0.30,center]{exptFourScreenShot/fig8.png}
		\caption{Configuring Hadoop MapReduce.}
		\label{fig:8}
	\end{figure}
	
	\begin{figure}[h]
		\includegraphics[scale=0.30,center]{exptFourScreenShot/fig9.png}
		\caption{Configuring Hadoop HDFS Site.}
		\label{fig:9}
	\end{figure}

 \pagebreak

\begin{figure}[h]
	\includegraphics[scale=0.35,center]{exptFourScreenShot/startall.png}
	\caption{Starting hadoop NameNode,Datanode,JobTracker and TaskTracker.}
	\label{fig:11}
\end{figure}

\begin{figure}[h]
	\includegraphics[scale=0.55,center]{exptFourScreenShot/fig12.png}
	\caption{Showing hadoop NameNode,Datanode,JobTracker and TaskTracker Processes.}
	\label{fig:12}
\end{figure}

\newpage
\begin{figure}[h]
	\includegraphics[scale=0.40,center]{exptFourScreenShot/fig13.png}
	\caption{Stopping hadoop NameNode,DataNode,JobTracker and TaskTracker.}
\end{figure}


\section{Result}
Thus the hadoop single node cluster is sucessfully created in Ubuntu 16.04 OS version and required packages are installed.

\end{document}